%%%%%%%%%%%%%%%%%%%%%%%%%%%%% Define Article %%%%%%%%%%%%%%%%%%%%%%%%%%%%%%%%%%
\documentclass[11pt,hidelinks]{book}
%%%%%%%%%%%%%%%%%%%%%%%%%%%%%%%%%%%%%%%%%%%%%%%%%%%%%%%%%%%%%%%%%%%%%%%%%%%%%%%

%%%%%%%%%%%%%%%%%%%%%%%%%%%%% Using Packages %%%%%%%%%%%%%%%%%%%%%%%%%%%%%%%%%%
\usepackage{geometry}
\usepackage{graphicx}
\usepackage{amssymb}
\usepackage{amsmath}
\usepackage{enumitem}
\usepackage{amsthm}
\usepackage{empheq}
\usepackage{mdframed}
\usepackage{booktabs}
\usepackage{changepage}
\usepackage{lipsum}
\usepackage{graphicx}
\usepackage{color}
\usepackage{hyperref}
\usepackage{psfrag}
\usepackage{pgfplots}
\usepackage[french]{babel}
\usepackage{bm}
%%%%%%%%%%%%%%%%%%%%%%%%%%%%%%%%%%%%%%%%%%%%%%%%%%%%%%%%%%%%%%%%%%%%%%%%%%%%%%%

% Other Settings

%%%%%%%%%%%%%%%%%%%%%%%%%% Page Setting %%%%%%%%%%%%%%%%%%%%%%%%%%%%%%%%%%%%%%%
\geometry{margin=1.8cm,head=14.5pt}
%%%%%%%%%%%%%%%%%%%%%%%%%% Define some useful colors %%%%%%%%%%%%%%%%%%%%%%%%%%
\definecolor{ocre}{RGB}{243,102,25}
\definecolor{mygray}{RGB}{243,243,244}
\definecolor{deepGreen}{RGB}{26,111,0}
\definecolor{shallowGreen}{RGB}{235,255,255}
\definecolor{deepBlue}{RGB}{61,124,222}
\definecolor{shallowBlue}{RGB}{235,249,255}
\definecolor{deepRed}{RGB}{133,1,1}
\definecolor{shallowRed}{RGB}{255,127,127}
\definecolor{lilac}{HTML}{c8a2c8}
\definecolor{deepPurple}{HTML}{643a64}
%%%%%%%%%%%%%%%%%%%%%%%%%%%%%%%%%%%%%%%%%%%%%%%%%%%%%%%%%%%%%%%%%%%%%%%%%%%%%%%

%%%%%%%%%%%%%%%%%%%%%%%%%% Define an orangebox command %%%%%%%%%%%%%%%%%%%%%%%%
\newcommand\orangebox[1]{\fcolorbox{ocre}{mygray}{\hspace{1em}#1\hspace{1em}}}
%%%%%%%%%%%%%%%%%%%%%%%%%%%%%%%%%%%%%%%%%%%%%%%%%%%%%%%%%%%%%%%%%%%%%%%%%%%%%%%

%%%%%%%%%%%%%%%%%%%%%%%%%%%% English Environments %%%%%%%%%%%%%%%%%%%%%%%%%%%%%
\newtheoremstyle{mytheoremstyle}{3pt}{3pt}{\normalfont}{0cm}{\rmfamily\bfseries}{}{1em}{{\color{black}\thmname{#1}~\thmnumber{#2}}\thmnote{\,--\,#3}}
\newtheoremstyle{myproblemstyle}{3pt}{3pt}{\normalfont}{0cm}{\rmfamily\bfseries}{}{1em}{{\color{black}\thmname{#1}~\thmnumber{#2}}\thmnote{\,--\,#3}}
\theoremstyle{mytheoremstyle}
\newmdtheoremenv[everyline=true,linewidth=1pt,backgroundcolor=shallowGreen,linecolor=deepGreen,innertopmargin=1pt,leftmargin=1pt,innerleftmargin=20pt,innerrightmargin=20pt]{theorem}{Théorème}[section]
\theoremstyle{mytheoremstyle}
\newmdtheoremenv[everyline=true,linewidth=1pt,backgroundcolor=shallowRed,linecolor=deepRed,innertopmargin=1pt,leftmargin=1pt,innerleftmargin=20pt,innerrightmargin=20pt,]{prop}{Proposition}[section]
\theoremstyle{mytheoremstyle}
\newmdtheoremenv[everyline=true,linewidth=1pt,backgroundcolor=lilac,linecolor=deepPurple,innertopmargin=1pt,leftmargin=1pt,innerleftmargin=20pt,innerrightmargin=20pt,]{definition}{Définition}[section]
\theoremstyle{mytheoremstyle}
\newmdtheoremenv[everyline=true,linewidth=1pt,backgroundcolor=shallowBlue,linecolor=deepBlue,innertopmargin=1pt,leftmargin=1pt,innerleftmargin=20pt,innerrightmargin=20pt,]{ef}{EF}[section]
\theoremstyle{mytheoremstyle}
\newmdtheoremenv[everyline=true,linewidth=1pt,backgroundcolor=shallowBlue, linecolor=deepBlue,innertopmargin=1pt,leftmargin=1pt, innerleftmargin=10pt, innerrightmargin=10pt,]{st}{S}[section]
\theoremstyle{mytheoremstyle}
\newmdtheoremenv[everyline=true,linewidth=1pt,backgroundcolor=mygray,linecolor=black,leftmargin=1pt,innertopmargin=1pt,innerleftmargin=20pt,innerrightmargin=20pt,]{ex}{}[section]

\theoremstyle{mytheoremstyle}
\newmdtheoremenv[everyline=true,linewidth=1pt,backgroundcolor=shallowBlue,linecolor=deepBlue,innertopmargin=1pt,leftmargin=1pt,innerleftmargin=20pt,innerrightmargin=20pt,]{rmq}{Remarque}[section]

\theoremstyle{mytheoremstyle}
\newmdtheoremenv[everyline=true,linewidth=1pt,backgroundcolor=mygray,linecolor=black,innertopmargin=1pt,leftmargin=1pt,innerleftmargin=20pt,innerrightmargin=20pt,]{exe}{Exercice}[section]


\theoremstyle{myproblemstyle}
\newmdtheoremenv[everyline=true,linecolor=black,leftmargin=1pt,innerleftmargin=10pt,innerrightmargin=10pt,]{problem}{Problem}[section]
%%%%%%%%%%%%%%%%%%%%%%%%%%%%%%%%%%%%%%%%%%%%%%%%%%%%%%%%%%%%%%%%%%%%%%%%%%%%%%%


%%%%%%%%%%%%%%%%%%%%%%%%%%%%%%% Plotting Settings %%%%%%%%%%%%%%%%%%%%%%%%%%%%%
\usepgfplotslibrary{colorbrewer}
\pgfplotsset{width=8cm,compat=1.9}
%%%%%%%%%%%%%%%%%%%%%%%%%%%%%%%%%%%%%%%%%%%%%%%%%%%%%%%%%%%%%%%%%%%%%%%%%%%%%%%


%%%
% Required to support mathematical unicode
\usepackage[warnunknown, fasterrors, mathletters]{ucs}
\usepackage[utf8x]{inputenc}
\allowdisplaybreaks[1]
% Always typeset math in display style
\everymath{\displaystyle}

% Use a larger font size
\usepackage[fontsize=14pt]{scrextend}

% Standard mathematical typesetting packages
\usepackage{amsfonts, amsthm, amsmath, amssymb}
\usepackage{mathtools}  % Extension to amsmath

% Symbol and utility packages
\usepackage{cancel, textcomp}
\usepackage[mathscr]{euscript}
\usepackage[nointegrals]{wasysym}

% Extras
\usepackage{physics}  % Lots of useful shortcuts and macros
\usepackage{tikz-cd}  % For drawing commutative diagrams easily
\usepackage{color}  % Add some colour to life
\usepackage{microtype}  % Minature font tweaks

% Common shortcuts
\def\mbb#1{\mathbb{#1}}
\def\mfk#1{\mathfrak{#1}}
\def\mfc#1{\mathcal{#1}}


\def\bN{\mbb{N}}
\def\bC{\mbb{C}}
\def\bR{\mbb{R}}
\def\bQ{\mbb{Q}}
\def\bZ{\mbb{Z}}
\def\L{\mfc{L}^1(I,\bK)}
\def\Li#1{\mfc{L}^{#1}(I,\bK)}
\def\LI#1{\mfc{L}^1(#1,\bK)}
\def\ib#1{\int_{a}^{b} #1}
\def\ig#1{\int_{a}^{\infty} #1}
\def\bK{\mbb{K}}
\def\af{[a,\infty[}
\def\ab{[a,b[}
\def\abc{]a,b]}
\def\abd{]a,b[}
\def\x{$x \in \bR$}
\def\z{$z \in \bC$}
\def\n{$n \in \bN$}
\def\is#1{\sum_{n=0}^\infty #1}
\def\iss#1#2{\sum_{n=#1}^\infty #2}
\def\se{\sum a_n z^n}
\def\ser{\sum a_n t^n}
\def\seq#1{\sum a_n z_{#1}^n}
\def\seb#1{\sum #1_n z^n}
\def\fn{\forall n \in \bN,}
\def\born{l^{\infty}\left( \bC \right)}
\def\fef{\textbf{FEF}}
\def\ln{\lim_{n \to \infty}}
\def\bO{\mfc{O}}
% Sometimes helpful macros
\newcommand{\func}[3]{#1\colon#2\to#3}
\newcommand{\cvs}[2]{converge simplement sur $#1$ vers $#2$}
\newcommand{\cvu}[2]{converge uniformément sur $#1$ vers $#2$}
\newcommand{\ppl}[1]{par passage à la limite lorsque #1}
\newcommand{\ppln}[1]{par passage à la limite lorsque $n \to \infty$}
\newcommand{\de}[4]{\begin{cases}
    #1 & \text{si } #2 \\
    #3 & \text{si } #4
\end{cases}}
\newcommand{\deq}[3]{\begin{cases}
    #1 & \text{si } #2 \\
    #3 & \text{sinon}
\end{cases}}
\newcommand{\vfunc}[5]{
  \begin{align*}
    #1 \colon #2 &\to #3\\
    #4 &\mapsto #5.
  \end{align*}
}
\newcommand{\parenth}[1]{\left(#1\right)}
\newcommand\restr[2]{{% we make the whole thing an ordinary symbol
  \left.\kern-\nulldelimiterspace % automatically resize the bar with \right
  #1 % the function
  \vphantom{\big|} % pretend it's a little taller at normal size
  \right|_{#2} % this is the delimiter
  }}
%%


%%%%%%%%%%%%%%%%%%%%%%%%%%%%%%% Title & Author %%%%%%%%%%%%%%%%%%%%%%%%%%%%%%%%
\title{Math3 CM}
\author{Tapé par C. THOMAS}
%%%%%%%%%%%%%%%%%%%%%%%%%%%%%%%%%%%%%%%%%%%%%%%%%%%%%%%%%%%%%%%%%%%%%%%%%%%%%%%

\begin{document}
    \maketitle
    \tableofcontents


    \chapter{Fonctions de $\bR$ dans $\bR$}

    Soit $D \in \bR$, soit $f \in \bR^{D}$ 
    \section{Limite}
    \subsection{Adhérence}
    \begin{definition}
      On appelle adhérence de $D$ le plus petit ensemble fermé qui contient D. Noté $\bar{D}$ 
    \end{definition}
    \subsection{Limite}
    Soit $f$ définie sur D, Soit $a \in \bar{D}$, Soit $l \in \bR$ 
    \begin{definition}
      On dit que $f$ a pour limite $l$ quand $x$ tends vers $a$ si
      \begin{align*}
        \forall \varepsilon > 0, \exists \eta > 0 | |x-a| < \eta \Rightarrow |f(x) - l| < \varepsilon
      \end{align*}
    \end{definition}
    \subsection{Fonctions négligeables}
    \begin{definition}
      Soit $f,g \in \bR^{D}$ et $a \in \bar{D}$ on dit que $f = o_{a}(g)$ si $\dfrac{f(x)}{g(x)} \to_{a} 0$
    \end{definition}
    \begin{ex}
      en 0 on a 
      \begin{align}
        \dfrac{f(x)}{g(x)} &= \dfrac{x}{\sqrt{x}} \\ 
                           &\to_{0^{+}} 0 \\ 
        f = o_{O^{+}}(g)
      \end{align}
    \end{ex}

    \subsection{Croissance comparée}
    \begin{theorem}
      \textbf{Croissances Comparées}\newline
      Soient $(\alpha,\beta,\gamma) \in R^{+*}$ avec $\gamma > 1$ avec 
      \begin{align*}
        f : x &\mapsto (\log x)^{\alpha} \\ 
        g : x &\mapsto x^{\beta} \\ 
        h : x &\mapsto \gamma^{x}
      \end{align*} 
      alors on a 
      \begin{align*}
        g = o_{\infty}(f) \\ 
        h = o_{\infty}(g)
      \end{align*}
      c'est à dire 
      \begin{align*}
        \dfrac{(\log x)^\alpha}{x^\beta} &\to_{\infty} 0 \\ 
        \dfrac{x^{\beta}}{\gamma^x} &\to_{\infty} 0
      \end{align*}
    \end{theorem}

    \subsection{Fonctions Équivalentes}
    \begin{definition}
      Soit $f,g \in \bR^{D}$ et $a \in \bar{D}$ on dit que $f$ est équivalente à $g$ 
      quand $x$ tends vers $a$ si $\dfrac{f}{g} \to_{a} 1$.  

      On note $f \equiv_a g$ 
    \end{definition}
    \begin{ex}
      \begin{itemize}
        \item Un polynome est équivalent à son monôme de plus haut degrès (resp bas) quand $x$ tends vers $\infty$ (resp $0$)
        \item $sin x \equiv_{0} x$
        \item $ln(1+x) \equiv_{0} x$
      \end{itemize}
    \end{ex}
    \subsection{Opération sur les équivalents}
    Soient $f_1,g_1,f_2,g_2 \in \bR^D$ soit $a \in \bar{D}$ soit $\alpha \in \bR$ 
    si 
    \begin{align*}
      f_1 &\equiv_{a} g_1 \\ 
      f_2 &\equiv_{a} g_2 \\
    \end{align*}
    alors
    \begin{itemize}
      \item \begin{align*}
      f_1 \cdot f_2 &\equiv_{a} g_1 \cdot g_2 \\ 
      \dfrac{f_1}{f_2} &\equiv_{a} \dfrac{g_1}{g_2} \\
      f_1^\alpha &\equiv_{a} g_1^\alpha
    \end{align*}
    \item \begin{equation}
      f = o_{a} g \Rightarrow f + g \equiv_{a} g 
    \end{equation}
    \item Si $f \equiv_{a} g$ et $\lim_{x \to a} f(x) = l$ alors $\lim_{x \to a} g(x) = l$
    \item \begin{prop}
      Si $f \equiv_a g$ et $lim_a f \not= 1$ alors $ln f \equiv_a ln g$
      \begin{proof}
        \begin{align*}
          \dfrac{\log g(x)}{\log f(x)} - 1 &= \dfrac{\log g(x) - \log f(x)}{\log f(x)}
          &= \dfrac{\dfrac{\log g(x)}{\log f(x)}}{\log f(x)}
          &to_{a} 0
        \end{align*}
      \end{proof}
    \end{prop}
    Cas particulier où $l=1$ 
    \begin{ex}
      $f(x) = 1+x$ et $g(x) = 1 + \sqrt{x}$ on a bien $f \equiv_0 g$ et $f \to_0 1$ 
      on a aussi $\log f(x) = \log 1+x \equiv_0 x$ et $ln g(x) = ln 1+\sqrt{x} \equiv_0 \sqrt{x}$ et $x \not= \sqrt{x}$
    \end{ex}
  \end{itemize}
  \section{Continuité}
  \begin{definition}
    Soit $f$ définie sur un ouvert $D$ de $\bR$ et $a \in D$.  
    On dit que $f$ est continue en $a$ si et seulement si $lim_a f(x) = f(a)$.  
    On note $\mathcal{C}^0$ l'ensemble des fonctions continues, c'est un espace vectoriel.
  \end{definition}
  \section{Dérivabilité}
  \begin{definition}
    Soit $f$ définie sur un ouvert $D$ de $\bR$ et $a \in D$.  
    On dit que $f$ est dérivable en $a$ si et seulement si $lim_a \dfrac{f(x) - f(a)}{x - a}$ existe dans $\bR$.  
    On note $f'$ la fonction $x \mapsto \lim_a \dfrac{f(x) - f(a)}{x - a}$ définie sur l'ensemble des valeurs dérivables de $f$.
  \end{definition}
  \subsection{Dérivée successives}
  On peut ensuite étudier la dérivabilité des dérivées successives de f
\end{document}