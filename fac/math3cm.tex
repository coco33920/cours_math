%%%%%%%%%%%%%%%%%%%%%%%%%%%%% Define Article %%%%%%%%%%%%%%%%%%%%%%%%%%%%%%%%%%
\documentclass[11pt,colorlinks]{book}
%%%%%%%%%%%%%%%%%%%%%%%%%%%%%%%%%%%%%%%%%%%%%%%%%%%%%%%%%%%%%%%%%%%%%%%%%%%%%%%

%%%%%%%%%%%%%%%%%%%%%%%%%%%%% Using Packages %%%%%%%%%%%%%%%%%%%%%%%%%%%%%%%%%%
\usepackage{geometry}
\usepackage{graphicx}
\usepackage{amssymb}
\usepackage{amsmath}
\usepackage{enumitem}
\usepackage{amsthm}
\usepackage{empheq}
\usepackage{mdframed}
\usepackage{booktabs}
\usepackage{changepage}
\usepackage{lipsum}
\usepackage{graphicx}
\usepackage{color}
\usepackage{hyperref}
\usepackage{psfrag}
\usepackage{pgfplots}
\usepackage[french]{babel}
\usepackage{bm}
%%%%%%%%%%%%%%%%%%%%%%%%%%%%%%%%%%%%%%%%%%%%%%%%%%%%%%%%%%%%%%%%%%%%%%%%%%%%%%%

% Other Settings

%%%%%%%%%%%%%%%%%%%%%%%%%% Page Setting %%%%%%%%%%%%%%%%%%%%%%%%%%%%%%%%%%%%%%%
\geometry{margin=1.8cm,head=14.5pt}
%%%%%%%%%%%%%%%%%%%%%%%%%% Define some useful colors %%%%%%%%%%%%%%%%%%%%%%%%%%
\definecolor{ocre}{RGB}{243,102,25}
\definecolor{mygray}{RGB}{243,243,244}
\definecolor{deepGreen}{RGB}{26,111,0}
\definecolor{shallowGreen}{RGB}{235,255,255}
\definecolor{deepBlue}{RGB}{61,124,222}
\definecolor{shallowBlue}{RGB}{235,249,255}
\definecolor{deepRed}{RGB}{133,1,1}
\definecolor{shallowRed}{RGB}{255,127,127}
\definecolor{lilac}{HTML}{c8a2c8}
\definecolor{deepPurple}{HTML}{643a64}
%%%%%%%%%%%%%%%%%%%%%%%%%%%%%%%%%%%%%%%%%%%%%%%%%%%%%%%%%%%%%%%%%%%%%%%%%%%%%%%

%%%%%%%%%%%%%%%%%%%%%%%%%% Define an orangebox command %%%%%%%%%%%%%%%%%%%%%%%%
\newcommand\orangebox[1]{\fcolorbox{ocre}{mygray}{\hspace{1em}#1\hspace{1em}}}
%%%%%%%%%%%%%%%%%%%%%%%%%%%%%%%%%%%%%%%%%%%%%%%%%%%%%%%%%%%%%%%%%%%%%%%%%%%%%%%

%%%%%%%%%%%%%%%%%%%%%%%%%%%% English Environments %%%%%%%%%%%%%%%%%%%%%%%%%%%%%
\newtheoremstyle{mytheoremstyle}{3pt}{3pt}{\normalfont}{0cm}{\rmfamily\bfseries}{}{1em}{{\color{black}\thmname{#1}~\thmnumber{#2}}\thmnote{\,--\,#3}}
\newtheoremstyle{myproblemstyle}{3pt}{3pt}{\normalfont}{0cm}{\rmfamily\bfseries}{}{1em}{{\color{black}\thmname{#1}~\thmnumber{#2}}\thmnote{\,--\,#3}}
\theoremstyle{mytheoremstyle}
\newmdtheoremenv[everyline=true,linewidth=1pt,backgroundcolor=shallowGreen,linecolor=deepGreen,innertopmargin=1pt,leftmargin=1pt,innerleftmargin=20pt,innerrightmargin=20pt]{theorem}{Théorème}[section]
\theoremstyle{mytheoremstyle}
\newmdtheoremenv[everyline=true,linewidth=1pt,backgroundcolor=shallowRed,linecolor=deepRed,innertopmargin=1pt,leftmargin=1pt,innerleftmargin=20pt,innerrightmargin=20pt,]{prop}{Proposition}[section]
\theoremstyle{mytheoremstyle}
\newmdtheoremenv[everyline=true,linewidth=1pt,backgroundcolor=lilac,linecolor=deepPurple,innertopmargin=1pt,leftmargin=1pt,innerleftmargin=20pt,innerrightmargin=20pt,]{definition}{Définition}[section]
\theoremstyle{mytheoremstyle}
\newmdtheoremenv[everyline=true,linewidth=1pt,backgroundcolor=shallowBlue,linecolor=deepBlue,innertopmargin=1pt,leftmargin=1pt,innerleftmargin=20pt,innerrightmargin=20pt,]{ef}{EF}[section]
\theoremstyle{mytheoremstyle}
\newmdtheoremenv[everyline=true,linewidth=1pt,backgroundcolor=shallowBlue, linecolor=deepBlue,innertopmargin=1pt,leftmargin=1pt, innerleftmargin=10pt, innerrightmargin=10pt,]{st}{S}[section]
\theoremstyle{mytheoremstyle}
\newmdtheoremenv[everyline=true,linewidth=1pt,backgroundcolor=mygray,linecolor=black,leftmargin=1pt,innertopmargin=1pt,innerleftmargin=20pt,innerrightmargin=20pt,]{ex}{}[section]

\theoremstyle{mytheoremstyle}
\newmdtheoremenv[everyline=true,linewidth=1pt,backgroundcolor=shallowBlue,linecolor=deepBlue,innertopmargin=1pt,leftmargin=1pt,innerleftmargin=20pt,innerrightmargin=20pt,]{rmq}{Remarque}[section]

\theoremstyle{mytheoremstyle}
\newmdtheoremenv[everyline=true,linewidth=1pt,backgroundcolor=mygray,linecolor=black,innertopmargin=1pt,leftmargin=1pt,innerleftmargin=20pt,innerrightmargin=20pt,]{exe}{Exercice}[section]


\theoremstyle{myproblemstyle}
\newmdtheoremenv[everyline=true,linecolor=black,leftmargin=1pt,innerleftmargin=10pt,innerrightmargin=10pt,]{problem}{Problem}[section]
%%%%%%%%%%%%%%%%%%%%%%%%%%%%%%%%%%%%%%%%%%%%%%%%%%%%%%%%%%%%%%%%%%%%%%%%%%%%%%%


%%%%%%%%%%%%%%%%%%%%%%%%%%%%%%% Plotting Settings %%%%%%%%%%%%%%%%%%%%%%%%%%%%%
\usepgfplotslibrary{colorbrewer}
\pgfplotsset{width=8cm,compat=1.9}
%%%%%%%%%%%%%%%%%%%%%%%%%%%%%%%%%%%%%%%%%%%%%%%%%%%%%%%%%%%%%%%%%%%%%%%%%%%%%%%


%%%
% Required to support mathematical unicode
\usepackage[warnunknown, fasterrors, mathletters]{ucs}
\usepackage[utf8x]{inputenc}
\allowdisplaybreaks[1]
% Always typeset math in display style
\everymath{\displaystyle}

% Use a larger font size
\usepackage[fontsize=14pt]{scrextend}

% Standard mathematical typesetting packages
\usepackage{amsfonts, amsthm, amsmath, amssymb}
\usepackage{mathtools}  % Extension to amsmath

% Symbol and utility packages
\usepackage{cancel, textcomp}
\usepackage[mathscr]{euscript}
\usepackage[nointegrals]{wasysym}

% Extras
\usepackage{physics}  % Lots of useful shortcuts and macros
\usepackage{tikz-cd}  % For drawing commutative diagrams easily
\usepackage{color}  % Add some colour to life
\usepackage{microtype}  % Minature font tweaks

% Common shortcuts
\def\mbb#1{\mathbb{#1}}
\def\mfk#1{\mathfrak{#1}}
\def\mfc#1{\mathcal{#1}}


\def\bN{\mbb{N}}
\def\bC{\mbb{C}}
\def\bR{\mbb{R}}
\def\bQ{\mbb{Q}}
\def\bZ{\mbb{Z}}
\def\L{\mfc{L}^1(I,\bK)}
\def\Li#1{\mfc{L}^{#1}(I,\bK)}
\def\LI#1{\mfc{L}^1(#1,\bK)}
\def\ib#1{\int_{a}^{b} #1}
\def\ig#1{\int_{a}^{\infty} #1}
\def\bK{\mbb{K}}
\def\af{[a,\infty[}
\def\ab{[a,b[}
\def\abc{]a,b]}
\def\abd{]a,b[}
\def\x{$x \in \bR$}
\def\z{$z \in \bC$}
\def\n{$n \in \bN$}
\def\is#1{\sum_{n=0}^\infty #1}
\def\iss#1#2{\sum_{n=#1}^\infty #2}
\def\se{\sum a_n z^n}
\def\ser{\sum a_n t^n}
\def\seq#1{\sum a_n z_{#1}^n}
\def\seb#1{\sum #1_n z^n}
\def\fn{\forall n \in \bN,}
\def\born{l^{\infty}\left( \bC \right)}
\def\fef{\textbf{FEF}}
\def\ln{\lim_{n \to \infty}}
\def\bO{\mfc{O}}
% Sometimes helpful macros
\newcommand{\func}[3]{#1\colon#2\to#3}
\newcommand{\cvs}[2]{converge simplement sur $#1$ vers $#2$}
\newcommand{\cvu}[2]{converge uniformément sur $#1$ vers $#2$}
\newcommand{\ppl}[1]{par passage à la limite lorsque #1}
\newcommand{\ppln}[1]{par passage à la limite lorsque $n \to \infty$}
\newcommand{\de}[4]{\begin{cases}
    #1 & \text{si } #2 \\
    #3 & \text{si } #4
\end{cases}}
\newcommand{\deq}[3]{\begin{cases}
    #1 & \text{si } #2 \\
    #3 & \text{sinon}
\end{cases}}
\newcommand{\vfunc}[5]{
  \begin{align*}
    #1 \colon #2 &\to #3\\
    #4 &\mapsto #5.
  \end{align*}
}
\newcommand{\parenth}[1]{\left(#1\right)}
\newcommand\restr[2]{{% we make the whole thing an ordinary symbol
  \left.\kern-\nulldelimiterspace % automatically resize the bar with \right
  #1 % the function
  \vphantom{\big|} % pretend it's a little taller at normal size
  \right|_{#2} % this is the delimiter
  }}
%%


%%%%%%%%%%%%%%%%%%%%%%%%%%%%%%% Title & Author %%%%%%%%%%%%%%%%%%%%%%%%%%%%%%%%
\title{Math3 CM}
\author{Cours de L. PASQUEREAU \\ Tapé par C. THOMAS}
%%%%%%%%%%%%%%%%%%%%%%%%%%%%%%%%%%%%%%%%%%%%%%%%%%%%%%%%%%%%%%%%%%%%%%%%%%%%%%%

\begin{document}
    \maketitle
    \tableofcontents


    \chapter{Fonctions de $\bR$ dans $\bR$}

    Soit $D \in \bR$, soit $f \in \bR^{D}$ 
    \section{Limite}
    \subsection{Adhérence}
    \begin{definition}
      On appelle adhérence de $D$ le plus petit ensemble fermé qui contient D. Noté $\bar{D}$ 
    \end{definition}
    \subsection{Limite}
    Soit $f$ définie sur D, Soit $a \in \bar{D}$, Soit $l \in \bR$ 
    \begin{definition}
      On dit que $f$ a pour limite $l$ quand $x$ tends vers $a$ si
      \begin{align*}
        \forall \varepsilon > 0, \exists \eta > 0 | |x-a| < \eta \Rightarrow |f(x) - l| < \varepsilon
      \end{align*}
    \end{definition}
    \subsection{Fonctions négligeables}
    \begin{definition}
      Soit $f,g \in \bR^{D}$ et $a \in \bar{D}$ on dit que $f = o_{a}(g)$ si $\dfrac{f(x)}{g(x)} \to_{a} 0$
    \end{definition}
    \begin{ex}
      en 0 on a 
      \begin{align}
        \dfrac{f(x)}{g(x)} &= \dfrac{x}{\sqrt{x}} \\ 
                           &\to_{0^{+}} 0 \\ 
        f = o_{O^{+}}(g)
      \end{align}
    \end{ex}

    \subsection{Croissance comparée}
    \begin{theorem}
      \textbf{Croissances Comparées}\newline
      Soient $(\alpha,\beta,\gamma) \in R^{+*}$ avec $\gamma > 1$ avec 
      \begin{align*}
        f : x &\mapsto (\log x)^{\alpha} \\ 
        g : x &\mapsto x^{\beta} \\ 
        h : x &\mapsto \gamma^{x}
      \end{align*} 
      alors on a 
      \begin{align*}
        g = o_{\infty}(f) \\ 
        h = o_{\infty}(g)
      \end{align*}
      c'est à dire 
      \begin{align*}
        \dfrac{(\log x)^\alpha}{x^\beta} &\to_{\infty} 0 \\ 
        \dfrac{x^{\beta}}{\gamma^x} &\to_{\infty} 0
      \end{align*}
    \end{theorem}

    \subsection{Fonctions Équivalentes}
    \begin{definition}
      Soit $f,g \in \bR^{D}$ et $a \in \bar{D}$ on dit que $f$ est équivalente à $g$ 
      quand $x$ tends vers $a$ si $\dfrac{f}{g} \to_{a} 1$.  

      On note $f \equiv_a g$ 
    \end{definition}
    \begin{ex}
      \begin{itemize}
        \item Un polynome est équivalent à son monôme de plus haut degrès (resp bas) quand $x$ tends vers $\infty$ (resp $0$)
        \item $sin x \equiv_{0} x$
        \item $ln(1+x) \equiv_{0} x$
      \end{itemize}
    \end{ex}
    \subsection{Opération sur les équivalents}
    Soient $f_1,g_1,f_2,g_2 \in \bR^D$ soit $a \in \bar{D}$ soit $\alpha \in \bR$ 
    si 
    \begin{align*}
      f_1 &\equiv_{a} g_1 \\ 
      f_2 &\equiv_{a} g_2 \\
    \end{align*}
    alors
    \begin{itemize}
      \item \begin{align*}
      f_1 \cdot f_2 &\equiv_{a} g_1 \cdot g_2 \\ 
      \dfrac{f_1}{f_2} &\equiv_{a} \dfrac{g_1}{g_2} \\
      f_1^\alpha &\equiv_{a} g_1^\alpha
    \end{align*}
    \item \begin{equation}
      f = o_{a} g \Rightarrow f + g \equiv_{a} g 
    \end{equation}
    \item Si $f \equiv_{a} g$ et $\lim_{x \to a} f(x) = l$ alors $\lim_{x \to a} g(x) = l$
    \item \begin{prop}
      Si $f \equiv_a g$ et $\lim_a f \not= 1$ alors $\log f \equiv_a \log g$
      \begin{proof}
        \begin{align*}
          \dfrac{\log g(x)}{\log f(x)} - 1 &= \dfrac{\log g(x) - \log f(x)}{\log f(x)} \\
          &= \dfrac{\log \left(\dfrac{g(x)}{f(x)}\right)}{\log f(x)} && \text{or } f \equiv_a g \\
          &\to_{a} \dfrac{0}{f(a)} && \text{par passage à la limite car } \lim_a f \not= 1 \\
          &= 0
        \end{align*}
        Donc $\lim_{x \to a} \dfrac{\log f(x)}{\log g(x)} = 1$ donc $\log f \equiv_a \log g$
      \end{proof}
    \end{prop}
    Cas particulier où $l=1$ 
    \begin{ex}
      $f(x) = 1+x$ et $g(x) = 1 + \sqrt{x}$ on a bien $f \equiv_0 g$ et $f \to_0 1$ 
      on a aussi $\log f(x) = \log 1+x \equiv_0 x$ et $\log g(x) = \log 1+\sqrt{x} \equiv_0 \sqrt{x}$ et $x \not= \sqrt{x}$
    \end{ex}
  \end{itemize}
  \section{Continuité}
  \begin{definition}
    Soit $f$ définie sur un ouvert $D$ de $\bR$ et $a \in D$.  
    On dit que $f$ est continue en $a$ si et seulement si $\lim_{x \to a} f(x) = f(a)$.  
    On note $\mathcal{C}^0$ l'ensemble des fonctions continues, c'est un espace vectoriel.
  \end{definition}
  \section{Dérivabilité}
  \begin{definition}
    Soit $f$ définie sur un ouvert $D$ de $\bR$ et $a \in D$.  
    On dit que $f$ est dérivable en $a$ si et seulement si $\lim_{x \to a} \dfrac{f(x) - f(a)}{x - a}$ existe dans $\bR$.  
    On note $f'$ la fonction $a \mapsto \lim_{x \to a} \dfrac{f(x) - f(a)}{x - a}$ définie sur l'ensemble des valeurs dérivables de $f$.
  \end{definition}
  \subsection{Dérivée successives}
  On peut ensuite étudier la dérivabilité des dérivées successives de f

  \section{Développements Limités (DL)}
  \begin{definition}
    On appelle Développement Limité (DL) à l'ordre $n$ et au point $a \in I$ d'une fonction $f$ défini sur un interval ouvert $I$ de $\bR$, un polynome $P$ tel que
    \begin{align*}
      \deg P &= n \\ 
      f(x) &= P(x-a) + o_0((x-a)^n)
    \end{align*}
    C'est une propriété \textbf{locale} de $f$ en $a$ 
  \end{definition}
  \subsection{Taylor-Young}
  \begin{theorem}[Formule de Taylor-Young]


    Soit $f$ une fonction définie de $I$ dans $\bR$, $n$ fois dérivable, alors $f$ admet un $DL_n$ pour un point $a$ de la forme 
    \begin{equation*}
      f(x) = \sum_{k=0}^n \dfrac{f^{(k)}(a)}{k!} (x-a)^k + o((x-a)^n)
    \end{equation*} 
  \end{theorem}
  \begin{rmq}
    Dans la majorité des cas pratiques, on prend $a=0$ ce qui donne
    \begin{equation*}
      f(x) = \sum_{k=0}^n \dfrac{f^{(k)}(0)x^k}{k!} + o(x^n)
    \end{equation*}
  \end{rmq}
  \begin{ex}
    En exemple on prend $f = \exp$, $\exp \in \mathcal{C}^{\infty}$ et on a $\forall n \in \bN, f^{(n)} = \exp$ donc $\forall n \in \bN, f^{(n)}(0) = 1$
    donc d'après le théorème de Taylor-Young, $\forall n \in \bN, \exp$ admet un $DL_n$ de la forme 
    \begin{align*}
      \exp(x) &= \sum_{k=0}^n \dfrac{\exp^{(k)}(0)}{k!} x^k + o(x^n) \\ 
      \exp(x) &= \sum_{k=0}^n \dfrac{x^k}{k!} + o(x^n)
    \end{align*}
  \end{ex}

  \begin{rmq}
    La formule de Taylor-Young permet aussi de faire l'inverse, de trouver la valeur d'une dérivée en un point si l'on connaît le DL de la fonction. 
    \begin{ex}
      Un exemple pour la valeur en $0$ de la dérivée quatrième de $\dfrac{1}{1-x}$ 
      \begin{equation*}
        \dfrac{1}{1-x} = 1 + x + x^2 + x^3 + x^4 + o(x^4)
      \end{equation*}
      Et d'après Taylor-Young on a 
      \begin{equation*}
        \dfrac{1}{1-x} = \dfrac{f(0)}{1} + \dfrac{f'(0)}{1}x + \dfrac{f''(0)}{2}x^2 + \dfrac{f^{(3)}(0)}{3!}x^3 + \dfrac{f^{(4)}(0)}{4!}x^4 + o(x^4)
      \end{equation*}
      Or les deux DL sont égaux, donc les polynômes aussi, et donc par identification des coefficients on a
      \begin{equation*}
        \dfrac{f^{(4)}(0)}{4!} = 1
      \end{equation*}
      ce qui donne 
      \begin{align*}
        \dfrac{f^{(4)}(0)}{4!} &= 1 \\ 
        f^{(4)}(0) = 4! = 24
      \end{align*}
      On a donc la valeur de la dérivée quatrième en $O$ sans avoir à dériver la fonction.
    \end{ex}
    En pratique ça permet l'étude des dérivées en un point sur des fonctions bien plus complexes
  \end{rmq}

  \subsection{DL usuels}
  \begin{prop}
    Les développements limités usuels en 0 sont les suivants
    \begin{align*}
      e^x &= \sum_{k=0}^{n} \dfrac{x^k}{k!} + o(x^n) \\
      \sin x &= \sum_{k=0}^{n} \dfrac{(-1)^{k} x^{2k+1}}{(2k+1)!} + o(x^{2n+1}) \\ 
      \cos x &= \sum_{k=0}^{n} \dfrac{(-1)^{k} x^{2k}}{(2k)!} + o(x^{2n}) \\
      \dfrac{1}{1-x} &= \sum_{k=0}^{n} x^k + o(x^n) \\ 
      \dfrac{1}{1+x} &= \sum_{k=0}^{n} (-1)^k x^k + o(x^n) \\ 
      \log (1+x) &= \sum_{k=0}^n \dfrac{(-1)^k x^k}{k} + o(x^n) \\ 
      (1+x)^{\alpha} &= \sum_{k=0}^{n} \sigma_{\alpha}(k) x^k + o(x^n) && \text{avec} \\
      \alpha &\in \bR && \text{et} \\
      \sigma_{\alpha}(k) &= 
      \begin{dcases}
        1 ,& \text{si } k=0 \\ 
        \dfrac{\sum_{i=0}^{k-1} (\alpha - i)}{k!} ,& \text{sinon}
      \end{dcases}
    \end{align*}
  \end{prop}
  \begin{rmq}
    Les DL de fonctions paires (resp impaires) ne contiennent que des coefficients sur les degrès pairs (resp impairs)
    \begin{ex}
      Exemple, la fonction $\cos$ est paire
    \end{ex}
  \end{rmq}

  \subsection{Opération sur les DL}
  Sans perte de généralité, les DL sont ici en $0$  \newline


  Soit $P,Q \in R[X]$ et $f,g \in \bR^{I}$ tels que 
  \begin{align*}
    \deg P &= \deg Q = n \\ 
    f(x) &= P(x) + o(x^n) \\ 
    g(x) &= Q(x) + o(x^n)
  \end{align*}
  \subsubsection{Troncage}
  \begin{definition}
    On appelle "troncage" à l'ordre $k \leq n$ d'un DL, le polynome tronqué $F_k$ de degrès $k$ tel que 
    tous les coefficients de $F_k$ sont égaux à ceux de $F$ jusqu'au coefficient de $x^k$ et tel que 
    \begin{equation*}
      f(x) = F_k(x) + o(x^k)
    \end{equation*}
  \end{definition}
  \begin{ex}
    On a 
    \begin{equation*}
      e^x = 1 + x + \dfrac{x^2}{2} + \dfrac{x^3}{3!} + \dfrac{x^4}{4!} + \dfrac{x^5}{5!} + o(x^5)
    \end{equation*}
    le $DL_5$ de $exp$ alors on peut le "tronquer" à l'ordre $k=3\leq 5$ pour avoir le $DL_3$ de exp 
    \begin{equation*}
      e^x = 1 + x + \dfrac{x^2}{2} + \dfrac{x^3}{3!} + o(x^3)
    \end{equation*}
  \end{ex}
  \subsubsection{Somme}
  \begin{prop}
    Le $DL_n$ de la fonction $f+g$ est la somme des $DL_n$ de $f$ et de $g$ 
    \begin{equation*}
      (f+g)(x) = P(x)+Q(x) + o(x^n)
    \end{equation*}
  \end{prop}
  \subsubsection{Produit}
  \begin{prop}
    Le $DL_n$ de la fonction $fg$ est le produit des $DL_n$ de $f$ et de $g$ tronqué à l'ordre $n$ 
    \begin{equation*}
      (fg)(x) = PQ_{n}(x) + o(x^n)
    \end{equation*}
  \end{prop}
  \subsubsection{Composée}
  \begin{prop}
    Si $g(0) = 0$ alors on peut composer les $DL_n$ et le $DL_n$ de $f \circ g$ est la composition des $DL_n$ de $f$ et de $g$ tronqué à l'ordre $n$
    \begin{equation*}
      (f\circ g)(x) = (P \circ Q)_{n}(x) + o(x^n)
    \end{equation*}
  \end{prop}
  \begin{ex}
    Exemple $DL_3$ de $\sqrt{1 + \sin x}$. On a bien $\sin 0 = 0$.
    \begin{align*}
      \sin x &= x - \dfrac{x^3}{6} + o(x^3) \\ 
      (1+X)^{\alpha} &= 1 + \alpha X + \dfrac{\alpha(\alpha-1)x^2}{2} X^2 + \dfrac{\alpha(\alpha-1)(\alpha-2)}{6} X^3 + o(X^3) && \text{donc} \\ 
      (1 + \sin x)^{\frac{1}{2}} &= 1 + \dfrac{1}{2} \left(x - \dfrac{x^3}{6}\right) - \dfrac{1}{8} \left(x - \dfrac{x^3}{6}\right)^2 + \dfrac{3}{48} \left(x - \dfrac{x^3}{6}\right)^3 + o(x^9) \\ 
      (1 + \sin x)^{\frac{1}{2}} &= 1 + \dfrac{1}{2} x - \dfrac{x^3}{12} - \dfrac{1}{8} x^2 + \dfrac{3}{48} x^3 + o(x^3) && \text{tronquage} \\ 
      (1 + \sin x)^{\frac{1}{2}} &= 1 + \dfrac{1}{2} x - \dfrac{1}{8} x^2 - \dfrac{1}{48} x^3 + o(x^3) 
    \end{align*}
  \end{ex}
  \subsection{Application au calcul de dérivé}
  Les DL sont utiles pour résoudre des formes indéterminées lors du calcul de limite 
  \begin{ex}
    Calcul de la limite en 0 de la fonction $f : $ $x \mapsto \dfrac{e^{x^2} - \cos x}{x^2}$  

    On calcule les différents DL à l'ordre 4
    \begin{align*}
      e^{x^2} &= 1 + (x^2) + \dfrac{(x^2)^2}{2} + o(x^4) \\
      \cos x &= 1 - \dfrac{x^2}{2} + \dfrac{x^4}{24} + o(x^4) \\
      e^{x^2} - \cos x &= \dfrac{3}{2}x^2 + o(x^2) && \text{tronquage, inutile au delà} \\ 
      f(x) &= \dfrac{\dfrac{3}{2}x^2 + o(x^2)}{x^2} \\ 
      f(x) &= \dfrac{3}{2} + o(1) && \text{d'où} \\
      \lim_{x \to 0} f(x) &= \dfrac{3}{2}
    \end{align*}
    On voit après que l'ordre 2 aurait suffit, l'intuition peut aider pour savoir à quel ordre calculer.
  \end{ex}
\end{document}